% wyklad.tex
\documentclass{beamer}
\usetheme{Amsterdam}

\usepackage[utf8]{inputenc}
\usepackage[polish]{babel}
\usepackage{polski}
\usepackage{listings}
\usepackage{url}
\setcounter{tocdepth}{2}

\setbeamercovered{dynamic}

% strona tytuowa
\title[Bit - Sekcja Grafiki]{Bit - Sekcja Grafiki}
\subtitle[Wprowadzanie do Direct3D]{Wprowadzanie do Direct3D}
\author[Dawid Fatyga]{Dawid Fatyga\\ \texttt{fatyga@student.agh.edu.pl}}
\institute[AGH]{Akademia Górniczo Hutnicza}
\date[Grudzień 2009]{Grudzień 3, 2009}

\lstset{language=C++}

\begin{document}

\begin{frame}
  \titlepage
\end{frame}

\begin{frame}
  \frametitle{O tym co dzisiaj}
  \tableofcontents[pausesections]
\end{frame}

\section{Powtórka z matematyki}
\subsection{Poruszane zagadnienia}
\begin{frame}
  \frametitle{Powtórka z matematyki}

  \begin{enumerate}
    \item Trochę o wektorach
    \item Macierze, wyznacznik i działania na macierzach
    \item Transformacje macierzowe
  \end{enumerate}

\end{frame}

\subsection{Trochę o wektorach}
\subsubsection{Definicja}

\begin{frame}
  \frametitle{Definicja}

  \begin{definition}[Wektor]
    Obiekt geometryczny w matematyce elementarnej, mający moduł (zwany też długością), kierunek i zwrot określający orientację wzdłuż danego kierunku.
  \end{definition}

\end{frame}

\begin{frame}
  \frametitle{Reprezetancja}
    \begin{definition}
  W trójwymiarowej przestrzeni euklidesowej (lub $ \mathbb{R}^3 $) wektory reprezentowane są jako trójki liczb odpowiadającym współrzędnym kartezjańskim punktu końcowego, co można zapisać: \\
  \begin{center}
    \begin{math}
      \overrightarrow{v} =
      \begin{bmatrix}
        v_{x} \\
        v_{y} \\
        v_{z}
      \end{bmatrix}
    \end{math}
  \end{center}
  \end{definition}

  \pause

  \begin{definition}[Długość wektora]
    \begin{math}
      |\overrightarrow{v}| = \sqrt{(v_{x}^2 + v_{y}^2 + v_{z}^2)}
    \end{math}
  \end{definition}

\end{frame}

\subsubsection{Działania na wektorach}
\begin{frame}
  \frametitle{Działania na wektorach}
  \begin{definition}[Iloczyn skalarny]
    Jest to pewne działanie przyporządkowujące parze wektorów pewną wartość rzeczywistą (skalarną). \\
    \begin{center}
      \begin{math}
        \overrightarrow{v} \cdot \overrightarrow{u} = v_{x}*u_{x} + v_{y}*u_{y} + v_{z}*u_{z}
      \end{math}
    \end{center}
  \end{definition}

  \pause

  \begin{definition}[Interpretacja geometryczna iloczynu skalarnego]
    Wartość iloczynu wektorowego jest równa iloczynowi długości tych wektorów i kosinusa konta między nimi: \\

  \begin{center}
    \begin{math}
      \overrightarrow{v} \cdot \overrightarrow{u} = |\overrightarrow{v}||\overrightarrow{u}|\cos(\overrightarrow{v}, \overrightarrow{u})
    \end{math}
  \end{center}
  \end{definition}
\end{frame}

\begin{frame}
  \frametitle{Iloczyn wektorowy}
  \begin{definition}
    Jest to pewne działanie przyporządkowujące parze wektorów $\overrightarrow{v}$ i $\overrightarrow{u}$ pewien wektor $\overrightarrow{w}$:
    \begin{enumerate}
      \item Jeśli $\overrightarrow{v}$ i $\overrightarrow{u}$ są liniowo zależne wynikiem jest wektor zerowy.
      \item W przeciwnym wypadku:
        \begin{itemize}
          \item $\overrightarrow{w} \perp \overrightarrow{v} \wedge \overrightarrow{w} \perp \overrightarrow{u}$
          \item $ |\overrightarrow{w}| = |\overrightarrow{v}||\overrightarrow{u}|\sin(\overrightarrow{v}, \overrightarrow{u}) $
        \end{itemize}
    \end{enumerate}
  \end{definition}

  O tym jak go wyznaczyć za chwilę...
\end{frame}

\subsection{Macierze, wyznacznik i działania na macierzach}
\subsubsection{Definicja}

\begin{frame}
  \frametitle{Definicja}
  \begin{definition}[Macierz]
    Układ zapisanych w postaci prostokątnej tablicy danych nazywanych elementami bądź współczynnikami. Dla nas szczególnie interesującę bedą macierze kwadratowe o wymiarach 4x4, dla przykładu: \\
     \begin{center}
    \begin{math}
      \textbf{M} =
      \begin{bmatrix}
        1  & 2  & 3  & 4 \\
        5  & 6  & 7  & 8 \\
        9  & 10 & 11 & 12 \\
        13 & 14 & 15 & 16
      \end{bmatrix}
    \end{math}
  \end{center}
  \end{definition}
\end{frame}

\subsubsection{Wyznacznik macierzy}

\begin{frame}
  \frametitle{Wyznacznik macierzy}
  \begin{definition}
    Operacja przyporządkowująca każdej macierzy prostokątnej pewną wartość rzeczywistą.
    \begin{center}
      \begin{math}
        \det{M} = \sum_{k = 0}^{N} (-1)^{1 + k} a_{ik} \det{M_{i, k}}
      \end{math}
    \end{center}
  \end{definition}
\end{frame}

\begin{frame}
  \frametitle{Wyznacznik macierzy a iloczyn wektorowy}
  \begin{definition}
    \begin{center}
      \begin{math}
        \overrightarrow{v} \times \overrightarrow{u} = \det{
          \begin{bmatrix}
        \hat{i}  & \hat{j}  & \hat{k} \\
        v_{x}  & v_{y} & v_{z} \\
        v_{x}  & u_{y} & u_{z}
          \end{bmatrix}
        } = \hat{i}(v_{y}u_{z} - v_{z}u_{y}) + \hat{j}(v_{x}u_{z} - v_{z}u_{x}) + \hat{k}(v_{x}u_{y} - v_{y}u_{x})
      = \begin{bmatrix}
        v_{y}u_{z} - v_{z}u_{y} \\
        v_{x}u_{z} - v_{z}u_{x} \\
        v_{x}u_{y} - v_{y}u_{x}
      \end{bmatrix} = \overrightarrow{w}
      \end{math}
    \end{center}
  \end{definition}
\end{frame}

\subsection{Działania na macierzach}
\begin{frame}
  \frametitle{Działania na macierzach}
  \begin{enumerate}
    \item<1-> Transpozycja
    \item<2-> Dodawanie
    \item<3-> Mnożenie
  \end{enumerate}
\end{frame}

\subsubsection{Transformacje macierzowe}
\begin{frame}
  \frametitle{Transformacje macierzowe}
  \begin{enumerate}
    \item<1-> Translacja
    \item<2-> Skalowanie
    \item<3-> Obrót
  \end{enumerate}
\end{frame}

\section{Wprowadzenie do Direct3D}
\subsection{Poruszane zagadnienia}
\begin{frame}
  \frametitle{Wprowadzenie do Direct3D}

  \begin{enumerate}
    \item Pierwszy program
    \item Gdzie, jak i co mam rysować?
    \item Trzeci wymiar
    \item Przekształcenia
    \item Teksturowanie
  \end{enumerate}

\end{frame}

\subsection{Pierwszy program w Direct3D}
\begin{frame}[fragile]
\frametitle{Nagłówki}
\begin{lstlisting}
#ifdef DEBUG
#define D3D_DEBUG_INFO
  #pragma comment (lib, "d3dx9d.lib")
#else
  #pragma comment (lib, "d3dx9.lib")
#endif // DEBUG
#pragma comment (lib, "d3d9.lib")
#include <d3dx9.h>
\end{lstlisting}
\end{frame}

\begin{frame}[fragile]
\frametitle{Tworzenie urządzenia}
\begin{lstlisting}
IDirect3D9* d3d = Direct3DCreate9(D3D_SDK_VERSION); // "obiekt" biblioteki
// ... parametry
IDirect3DDevice9* device;
HRESULT result = d3d->CreateDevice(
                  D3DADAPTER_DEFAULT,
                  D3DDEVTYPE_HAL,
                  app.window_handle(),
   D3DCREATE_SOFTWARE_VERTEXPROCESSING, // ;(
                  &parameters,
                  &device);
if(FAILED(result)) return 1; // Smutno :(
\end{lstlisting}
\end{frame}

\begin{frame}[fragile]
\frametitle{Parametry urządzenia}
\begin{lstlisting}
// ... parametry
D3DPRESENT_PARAMETERS parameters;
ZeroMemory(&parameters, sizeof(parameters));
parameters.Windowed = true;
\end{lstlisting}
\begin{itemize}
  \item \texttt{Windowed} -- okno czy na pełny ekran?
  \item \texttt{SwapEffect} -- zalecane \texttt{D3DSWAPEFFECT\_DISCARD}
  \item \texttt{BackBufferCount} -- ilość dodatkowych buforów
  \item \texttt{BackBufferFormat} -- \texttt{D3DFMT\_X8R8G8B8}
  \item \texttt{EnableAutoDepthStencil} -- użyć bufora głębi?
  \item \texttt{BackBufferFormat} -- \texttt{D3DFMT\_D16} albo \texttt{D3DFMT\_D24S8}
\end{itemize}
\end{frame}

\begin{frame}[fragile]
\frametitle{Główna pętla i sprzątanie}
\begin{lstlisting}
while(app.running()){
  device->Clear(0, NULL,
      D3DCLEAR_TARGET,
      D3DCOLOR_XRGB(0, 0, 0),
      1.0f, 0);
  device->BeginScene();
  // ...
  device->EndScene();
  device->Present(NULL, NULL, NULL, NULL);
}
device->Release();
d3d->Release();
\end{lstlisting}

\end{frame}

\subsection{Gdzie, jak i co mam rysować?}
\begin{frame}[fragile]
\frametitle{Wierzchołki}

\begin{lstlisting}
#define Vertex_Format \
  (D3DFVF_XYZRHW | D3DFVF_DIFFUSE)
struct Vertex {
  float x, y, z, rhw;
  DWORD color;
};
\end{lstlisting}

\begin{itemize}
  \item \texttt{D3DFVF\_XYZRHW} -- współrzędne do rysowania od razu na ekranie
  \item \texttt{D3DFVF\_XYZ} -- współrzędne do transformacji
  \item \texttt{D3DFVF\_DIFFUSE} -- kolor wierzchołka
  \item \texttt{D3DFVF\_TEX1} -- współrzędne tekstury
\end{itemize}

\end{frame}

\begin{frame}[fragile]
\frametitle{Tablice wierzchołków}

\begin{lstlisting}
Vertex triangle[] = {
// X    Y    Z  W  Color
  {400, 100, 1, 1, D3DCOLOR_XRGB(0, 0, 255)},
  {600, 500, 1, 1, D3DCOLOR_XRGB(0, 255, 0)},
  {200, 500, 1, 1, D3DCOLOR_XRGB(255, 0, 0)}
};
\end{lstlisting}

\end{frame}

\begin{frame}[fragile]
\frametitle{Rysujemy}

\begin{lstlisting}
device->SetFVF(Vertex_Format);
device->Clear(0, NULL, D3DCLEAR_TARGET,
      D3DCOLOR_XRGB(0, 0, 0), 1.0f, 0);

device->BeginScene();
device->DrawPrimitiveUP(D3DPT_TRIANGLESLIST, 1,
                    triangle, sizeof(Vertex));
device->EndScene();
device->Present(NULL, NULL, NULL, NULL);
\end{lstlisting}

\end{frame}

\begin{frame}[fragile]
\frametitle{Bufor wierzchołków}

Jest jedynym słuszym sposobem rysowania czegokolwiek.

\begin{lstlisting}
IDirect3DVertexBuffer9* buffer;
device->CreateVertexBuffer(3 * sizeof(Vertex), 0,
    Vertex_Format,
    D3DPOOL_MANAGED,
    &buffer,
    NULL);
// a po wywolaniu device->BeginScene() ...
device->SetStreamSource(0, buffer,
    0, sizeof(Vertex));
device->DrawPrimitive(D3DPT_TRIANGLELIST, 0, 1);
\end{lstlisting}

\end{frame}

\begin{frame}
\frametitle{Rodzaje rysowania trójkątów}

\begin{itemize}
  \item \texttt{D3DPT\_TRIANGLELIST} -- Lista trójkątów
  \item \texttt{D3DPT\_TRIANGLESTRIP} -- Nowy wierzchołek tworzy trójkąt z dwoma poprzednimi
  \item \texttt{D3DPT\_TRIANGLESFAN} -- Nowy wierzchołek tworzy trójkąt z poprzednim i pierwszym wierzchołkiem (pierwszy wierchołek jest centrum ''wachlarza'')
\end{itemize}

\end{frame}

\subsection{Trzeci wymiar i teksturowanie}
\begin{frame}[fragile]
\frametitle{Trzeci wymiar -- czego potrzebujemy?}

\begin{itemize}
  \item<1-> Innych wierzchołków -- \texttt{D3DFVF\_XYZ}
  \item<2-> Przekształcenia projekcji -- perspektywiczne / ortogonalne.
  \item<3-> Przekształcenia widoku, czyli oka kamery.
  \item<4-> Modelu do rysowania -- przykładowo pudełka.
\end{itemize}

\end{frame}

\begin{frame}[fragile]
\frametitle{Kiedy kolory nam się nudzą...}

\begin{itemize}
  \item<1-> Nowa flaga w formacie wierzchołków -- \texttt{D3DFVF\_TEX1}
  \item<2-> Współrzędne tekstury.
  \item<3-> Obiekty tekstur.
\end{itemize}

\end{frame}

\begin{frame}[fragile]
\frametitle{Tekstury w pigułce}

\begin{itemize}
  \item<1-> Najprościej:
\begin{lstlisting}
LPDIRECT3DTEXTURE9 texture;
D3DXCreateTextureFromFile(device,
  "images\\texture.jpg",
  &texture);
\end{lstlisting}
  \item<2-> \texttt{D3DXCreateTextureFromFileEx()} daje większe możliwości.
  \item<3-> Włączamy teksturę za pomocą:
\begin{lstlisting}
device->SetTexture(0, texture)
\end{lstlisting}
  \item<4-> Usuwamy standardowo za pomocą \texttt{texture->Release()}
\end{itemize}

\end{frame}


\begin{frame}[fragile]
\frametitle{Odnośniki}

\begin{itemize}
  \item \texttt{www.chadvernon.com/blog/tutorials/directx9/}
  \item \texttt{www.two-kings.de/tutorials/dxgraphics/}
  \item \texttt{www.danielloran.com/study/directx/Default.aspx}
  \item \texttt{www.codesampler.com/dx9src.htm}
\end{itemize}
\end{frame}

\end{document}

