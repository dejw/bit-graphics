% wyklad.tex
\documentclass{beamer}
\usetheme{Amsterdam}

\usepackage[utf8]{inputenc}
\usepackage[polish]{babel}
\usepackage{polski}
\usepackage{listings}
\usepackage{url}
\setcounter{tocdepth}{2}

\setbeamercovered{dynamic}

% strona tytuowa
\title[Bit - Sekcja Grafiki]{Bit - Sekcja Grafiki}
\subtitle[Wprowadzanie do Direct3D]{Wprowadzanie do Direct3D}
\author[Dawid Fatyga]{Dawid Fatyga\\ \texttt{fatyga@student.agh.edu.pl}}
\institute[AGH]{Akademia Górniczo Hutnicza}
\date[Grudzień 2009]{Grudzień 3, 2009}

\begin{document}

\begin{frame}
  \titlepage
\end{frame}

\begin{frame}
  \frametitle{O tym co dzisiaj}
  \tableofcontents[pausesections]
\end{frame}

\section{Powtórka z matematyki}
\subsection{Poruszane zagadnienia}
\begin{frame}
  \frametitle{Powtórka z matematyki}

  \begin{enumerate}
    \item Trochę o wektorach
    \item Macierze, wyznacznik i działania na macierzach
    \item Transformacje macierzowe
  \end{enumerate}

\end{frame}

\subsection{Trochę o wektorach}
\subsubsection{Definicja}

\begin{frame}
  \frametitle{Definicja}

  \begin{definition}[Wektor]
    Obiekt geometryczny w matematyce elementarnej, mający moduł (zwany też długością), kierunek i zwrot określający orientację wzdłuż danego kierunku.
  \end{definition}

\end{frame}

\begin{frame}
  \frametitle{Reprezetancja}
    \begin{definition}
  W trójwymiarowej przestrzeni euklidesowej (lub $ \mathbb{R}^3 $) wektory reprezentowane są jako trójki liczb odpowiadającym współrzędnym kartezjańskim punktu końcowego, co można zapisać: \\
  \begin{center}
    \begin{math}
      \overrightarrow{v} =
      \begin{bmatrix}
        v_{x} \\
        v_{y} \\
        v_{z}
      \end{bmatrix}
    \end{math}
  \end{center}
  \end{definition}

  \pause

  \begin{definition}[Długość wektora]
    \begin{math}
      |\overrightarrow{v}| = \sqrt{(v_{x}^2 + v_{y}^2 + v_{z}^2)}
    \end{math}
  \end{definition}

\end{frame}

\subsubsection{Działania na wektorach}
\begin{frame}
  \frametitle{Działania na wektorach}
  \begin{definition}[Iloczyn skalarny]
    Jest to pewne działanie przyporządkowujące parze wektorów pewną wartość rzeczywistą (skalarną). \\
    \begin{center}
      \begin{math}
        \overrightarrow{v} \cdot \overrightarrow{u} = v_{x}*u_{x} + v_{y}*u_{y} + v_{z}*u_{z}
      \end{math}
    \end{center}
  \end{definition}

  \pause

  \begin{definition}[Interpretacja geometryczna iloczynu skalarnego]
    Wartość iloczynu wektorowego jest równa iloczynowi długości tych wektorów i kosinusa konta między nimi: \\

  \begin{center}
    \begin{math}
      \overrightarrow{v} \cdot \overrightarrow{u} = |\overrightarrow{v}||\overrightarrow{u}|\cos(\overrightarrow{v}, \overrightarrow{u})
    \end{math}
  \end{center}
  \end{definition}
\end{frame}

\begin{frame}
  \frametitle{Iloczyn wektorowy}
  \begin{definition}
    Jest to pewne działanie przyporządkowujące parze wektorów $\overrightarrow{v}$ i $\overrightarrow{u}$ pewien wektor $\overrightarrow{w}$:
    \begin{enumerate}
      \item Jeśli $\overrightarrow{v}$ i $\overrightarrow{u}$ są liniowo zależne wynikiem jest wektor zerowy.
      \item W przeciwnym wypadku:
        \begin{itemize}
          \item $\overrightarrow{w} \perp \overrightarrow{v} \wedge \overrightarrow{w} \perp \overrightarrow{u}$
          \item $ |\overrightarrow{w}| = |\overrightarrow{v}||\overrightarrow{u}|\sin(\overrightarrow{v}, \overrightarrow{u}) $
        \end{itemize}
    \end{enumerate}
  \end{definition}

  O tym jak go wyznaczyć za chwilę...
\end{frame}

\subsection{Macierze, wyznacznik i działania na macierzach}
\subsubsection{Definicja}

\begin{frame}
  \frametitle{Definicja}
  \begin{definition}[Macierz]
    Układ zapisanych w postaci prostokątnej tablicy danych nazywanych elementami bądź współczynnikami. Dla nas szczególnie interesującę bedą macierze kwadratowe o wymiarach 4x4, dla przykładu: \\
     \begin{center}
    \begin{math}
      \textbf{M} =
      \begin{bmatrix}
        1  & 2  & 3  & 4 \\
        5  & 6  & 7  & 8 \\
        9  & 10 & 11 & 12 \\
        13 & 14 & 15 & 16
      \end{bmatrix}
    \end{math}
  \end{center}
  \end{definition}
\end{frame}

\subsubsection{Wyznacznik macierzy}

\begin{frame}
  \frametitle{Wyznacznik macierzy}
  \begin{definition}
    Operacja przyporządkowująca każdej macierzy prostokątnej pewną wartość rzeczywistą.
    \begin{center}
      \begin{math}
        \det{M} = \sum_{k = 0}^{N} (-1)^{1 + k} a_{ik} \det{M_{i, k}}
      \end{math}
    \end{center}
  \end{definition}
\end{frame}

\begin{frame}
  \frametitle{Wyznacznik macierzy a iloczyn wektorowy}
  \begin{definition}
    \begin{center}
      \begin{math}
        \overrightarrow{v} \times \overrightarrow{u} = \det{
          \begin{bmatrix}
        \hat{i}  & \hat{j}  & \hat{k} \\
        v_{x}  & v_{y} & v_{z} \\
        v_{x}  & u_{y} & u_{z}
          \end{bmatrix}
        } = \hat{i}(v_{y}u_{z} - v_{z}u_{y}) + \hat{j}(v_{x}u_{z} - v_{z}u_{x}) + \hat{k}(v_{x}u_{y} - v_{y}u_{x})
      = \begin{bmatrix}
        v_{y}u_{z} - v_{z}u_{y} \\
        v_{x}u_{z} - v_{z}u_{x} \\
        v_{x}u_{y} - v_{y}u_{x}
      \end{bmatrix} = \overrightarrow{w}
      \end{math}
    \end{center}
  \end{definition}
\end{frame}

\subsection{Działania na macierzach}
\begin{frame}
  \frametitle{Działania na macierzach}
  \begin{enumerate}
    \item<1-> Transpozycja
    \item<2-> Dodawanie
    \item<3-> Mnożenie
  \end{enumerate}
\end{frame}

\subsubsection{Transformacje macierzowe}
\begin{frame}
  \frametitle{Transformacje macierzowe}
  \begin{enumerate}
    \item<1-> Translacja
    \item<2-> Skalowanie
    \item<3-> Obrót
  \end{enumerate}
\end{frame}

\section{Wprowadzenie do Direct3D}
\subsection{Poruszane zagadnienia}
\begin{frame}
  \frametitle{Wprowadzenie do Direct3D}

  \begin{enumerate}
    \item Tworzenie urządzenia
    \item Gdzie i jak mam rysować?
    \item Trzeci wymiar
    \item Przekształcenia
    \item Teksturowanie
  \end{enumerate}

\end{frame}

\end{document}

